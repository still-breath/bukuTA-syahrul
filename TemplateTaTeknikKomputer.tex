\title{Buku Tugas Akhir ITS}
\author{Musk, Elon Reeve}
% Pengaturan ukuran teks dan bentuk halaman dua sisi
\documentclass[12pt,twoside]{report}
% Pengaturan ukuran halaman dan margin
\usepackage[a4paper,top=30mm,left=30mm,right=20mm,bottom=25mm]{geometry}
\usepackage[table,xcdraw]{xcolor}

\usepackage[english]{babel}
% Pengaturan pewarnaan
\usepackage{float}
%\usepackage{colortbl} 
\usepackage[utf8]{inputenc}
\usepackage{amsmath}
\usepackage{multirow} % for multirow command
% Pengaturan ukuran spasi
\usepackage[singlespacing]{setspace}
\usepackage{rotating} % for sidewaystable environment
\usepackage[ruled,vlined]{algorithm2e} % for algorithm2e environment
\usepackage{pdflscape} % for landscape environment
\usepackage{hyphenat} % for \nohyphens command
% Pengaturan detail pada file PDF
\usepackage[pdfauthor={\@author},bookmarksnumbered,pdfborder={0 0 0}]{hyperref}

% Pengaturan jenis karakter
\usepackage[utf8]{inputenc}
\usepackage{makecell}
\usepackage{amsmath}
\usepackage{subcaption}


% Pengaturan kutipan artikel
%\usepackage[style=ieee, backend=biber]{biblatex}

% Package lainnya
\usepackage{changepage}
\usepackage{enumitem}
\usepackage{eso-pic}
\usepackage{txfonts} % Font times
\usepackage{etoolbox}
\usepackage{graphicx}
\usepackage{lipsum}
\usepackage{caption}
\usepackage{longtable}


\usepackage{tabularx}
\usepackage{wrapfig}
\usepackage{float}
\usepackage{array} % Add this line in the preamble
\usepackage{ifthen}
\usepackage{etoolbox}
\usepackage{cite}

% Add your bibliography file
%\addbibresource{pustaka/pustaka.bib}

% Define boolean variables

%\captionsetup[longtable]{
%  width=.9\textwidth, % Adjust the width to .9 of the text width or as needed
%}

% Definisi untuk "Hati ini sengaja dikosongkan"
\patchcmd{\cleardoublepage}{\hbox{}}{
  \thispagestyle{empty}
  \vspace*{\fill}
  \begin{center}\textit{[Halaman ini sengaja dikosongkan]}\end{center}
  \vfill}{}{}

% Pengaturan penomoran halaman
\usepackage{fancyhdr}
\fancyhf{}
\renewcommand{\headrulewidth}{0pt}
\pagestyle{fancy}
\fancyfoot[LE,RO]{\thepage}
\patchcmd{\chapter}{plain}{fancy}{}{}
\patchcmd{\chapter}{empty}{plain}{}{}

% Command untuk bulan
\newcommand{\MONTH}{%
  \ifcase\the\month
  \or Januari% 1
  \or Februari% 2
  \or Maret% 3
  \or April% 4
  \or Mei% 5
  \or Juni% 6
  \or Juli% 7
  \or Agustus% 8
  \or September% 9
  \or Oktober% 10
  \or November% 11
  \or Desember% 12
  \fi}
\newcommand{\ENGMONTH}{%
  \ifcase\the\month
  \or January% 1
  \or February% 2
  \or March% 3
  \or April% 4
  \or May% 5
  \or June% 6
  \or July% 7
  \or August% 8
  \or September% 9
  \or October% 10
  \or November% 11
  \or December% 12
  \fi}

% Pengaturan format judul bab
\usepackage{titlesec}
\titleformat{\chapter}[display]{\bfseries\Large}{BAB \centering\Roman{chapter}}{0ex}{\vspace{0ex}\centering}
\titleformat{\section}{\bfseries\large}{\MakeUppercase{\thesection}}{1ex}{\vspace{1ex}}
\titleformat{\subsection}{\bfseries\large}{\MakeUppercase{\thesubsection}}{1ex}{}
\titleformat{\subsubsection}{\bfseries\large}{\MakeUppercase{\thesubsubsection}}{1ex}{}
\titlespacing{\chapter}{0ex}{0ex}{4ex}
\titlespacing{\section}{0ex}{1ex}{0ex}
\titlespacing{\subsection}{0ex}{0.5ex}{0ex}
\titlespacing{\subsubsection}{0ex}{0.5ex}{0ex}
\setcounter{secnumdepth}{3} % Untuk memberi penomoran pada \subsubsection
\newcommand{\engtatitle}{dadfa}

\newcommand{\NamaMahasiswa}[2]{
	\newcommand{\name}{#1}
	\newcommand{\nrp}{#2}

}

\newcommand{\JudulTAInd}[1]{
	\newcommand{\tatitle}{#1}
}
\newcommand{\JudulTAEng}[1]{
			\renewcommand{\engtatitle}{#1}
	}
	
\newcommand{\Tempat}[1]{
		%Surabaya
		\newcommand{\place}{#1}

	}




\newbool{bpembimbing2}
\newbool{bpenguji3}
\setbool{bpembimbing2}{false}
\setbool{bpenguji3}{false}



\newcommand{\PembimbingSatu}[2]{
\newcommand{\advisor}{#1}
\newcommand{\advisornip}{#2}
} 
\newcommand{\PembimbingDua}[2]{
\newcommand{\coadvisor}{#1}
\newcommand{\coadvisornip}{#2}
\setbool{bpembimbing2}{true}
}


\newcommand{\PengujiSatu}[2]{
\newcommand{\examinerone}{#1}
\newcommand{\examineronenip}{#2}

} 

\newcommand{\PengujiDua}[2]{
\newcommand{\examinertwo}{#1}
\newcommand{\examinertwonip}{#2}

} 

\newcommand{\PengujiTiga}[2]{
\newcommand{\examinerthree}{#1}
\newcommand{\examinerthreenip}{#2}

\setbool{bpenguji3}{true}

}

\newcommand{\KepalaDepartemen}[2]{
\newcommand{\headofdepartment}{#1}
\newcommand{\headofdepartmentnip}{#2}
} 



% jurusan\
\newcommand{\Departemen}[2]{
\newcommand{\studyprogram}{#1}
\newcommand{\engstudyprogram}{#2}
\newcommand{\department}{#1}
\newcommand{\engdepartment}{#2}
}

% fakultas
\newcommand{\Fakultas}[2]{
\newcommand{\faculty}{#1}
\newcommand{\engfaculty}{#2}
}
% singkatan fakultas
\newcommand{\SingkatanFakultas}[2]{
\newcommand{\facultyshort}{#1}
\newcommand{\engfacultyshort}{#2}
}

% departemen

% kode mata kuliah
\newcommand{\KodeMataKuliah}[1]{
\newcommand{\coursecode}{#1}
}



%\input{pustaka/variables.tex}

\input{pustaka/tanda-hubung.tex}



% Pengaturan format potongan kode
\usepackage{listings}
\definecolor{comment}{RGB}{0,128,0}
\definecolor{string}{RGB}{255,0,0}
\definecolor{keyword}{RGB}{0,0,255}
\lstdefinestyle{codestyle}{
  commentstyle=\color{comment},
  stringstyle=\color{string},
  keywordstyle=\color{keyword},
  basicstyle=\footnotesize\ttfamily,
  numbers=left,
  numberstyle=\tiny,
  numbersep=5pt,
  frame=lines,
  breaklines=true,
  prebreak=\raisebox{0ex}[0ex][0ex]{\ensuremath{\hookleftarrow}},
  showstringspaces=false,
  upquote=true,
  tabsize=2,
}
\lstset{style=codestyle}
\usepackage{array}

%========================================
%Template ini di modifikasi oleh departement Teknik Komputer ITS
%dari template yang telah dibuat Musk, Elon Reeve
%=======================================

%=======================================
%Direktori Penting 
%1. Init : Berisi File penting terkait dengan konfigurasi .
%            Isi dari direktori ini jangan diubah
%2.KataPengantar 
%3. Abstrak : berisi abstrak untuk bahasa indonesia dan bahasa inggris 
%4. Bab1-Bab5 : Isi tiap-tiap bab
%5. Pustaka : Berisi daftar pustaka 
%6. Biografi penulis 
%=======================================

%========================================
% Identutas Mahasiswa} 
%========================================
\NamaMahasiswa{Syahrul Fathoni Ahmad}{5024 21 1007}

%========================================
%Identitas Tugas Akhir
%========================================
\KodeMataKuliah{EC234801}
\JudulTAInd{SISTEM ESTIMASI KECEPATAN KENDARAAN LALU
LINTAS MENGGUNAKAN DRONE DENGAN METODE \emph{YOLOv8} BERBASIS \emph{JETSON NANO}}
\JudulTAEng{VEHICLE SPEED ESTIMATION SYSTEM USING DRONE
WITH JETSON NANO-BASED YOLOv8 METHOD}
\Tempat{Surabaya}

%========================================
% Daftar Pembimbing 
%========================================
\PembimbingSatu{Dr. Arief Kurniawan, S.T., M.T.}{19740907 200212 1 001} 
\PembimbingDua{Dr. Diah Puspito Wulandari, S.T., M.Sc.}{19801219200501 2 001} 

%========================================
% Daftar Penguji
%========================================
\PengujiSatu{Dr. Eko Mulyanto Yuniarno,S.T.,M.T.}{19680601199512 1 009} 
\PengujiDua{Arta Kusuma Hernanda, S.T., M.T.}{1996202311024}
\PengujiTiga{Yusril Izza, S.T., M.Sc.}
{} 


%========================================
%Identitas Departement dan fakultas
%========================================
\KepalaDepartemen{Dr. Arief Kurniawan, S.T., M.T.}{19740907200212 1 001} 
\Departemen{Teknik Komputer}{Computer Engineering}
\Fakultas{Teknologi Elektro dan Informatika Cerdas}{Intelligent Electrical And Informatics Technology}
\SingkatanFakultas{FTEIC}{FIEI}


% Isi keseluruhan dokumen
\begin{document}

\AddToShipoutPictureBG*{
  \AtPageLowerLeft{
    % Ubah nilai berikut jika posisi horizontal background tidak sesuai
    \hspace{-3.25mm}
c
    % Ubah nilai berikut jika posisi vertikal background tidak sesuai
    \raisebox{0mm}{
      \includegraphics[width=\paperwidth,height=\paperheight]{Init/gambar/sampul-luar.png}
    }
  }
}

% Menyembunyikan nomor halaman
\thispagestyle{empty}

% Pengaturan margin untuk menyesuaikan konten sampul
\newgeometry{
  top=55mm,
  left=30mm,
  right=20mm,
  bottom=20mm
}

\begin{flushleft}

  % Pemilihan font sans serif
  \sffamily

  % Pemilihan warna font putih
  \color{white}

  % Pemilihan font bold
  \fontseries{bx}
  \selectfont
  \begin{spacing}{1.5}
   \begin{large}
  TUGAS AKHIR - \coursecode{}
\end{large}

\vspace{\fill}

\begin{spacing}{1.5}
  \begin{Large}
    \tatitle{}
  \end{Large}
\end{spacing}

\vspace{\fill}

\begin{large}
  \name{} \\
  \textmd{NRP \nrp{}}
\end{large}

\vspace{\fill}

\begin{large}
  \textmd{Dosen Pembimbing} \\
  \advisor{} \\
  \textmd{NIP \advisornip{}} \\
  \coadvisor{} \\
  \textmd{NIP \coadvisornip{}}
\end{large}

\vspace{\fill}

Program Studi Strata 1 (S1) \studyprogram{} \\

\mdseries

Departemen \department{} \\
Fakultas \faculty{} \\
Institut Teknologi Sepuluh Nopember

\place{} \\ \the\year{}

  \end{spacing}

\end{flushleft}

\restoregeometry

\cleardoublepage

% Atur ulang penomoran halaman
\setcounter{page}{1}

% Sampul dalam Bahasa Indonesia
\input{Init/sampul-luar-tipis.tex}
\clearpage
\cleardoublepage

% Sampul dalam Bahasa Inggris

\AddToShipoutPictureBG*{
  \AtPageLowerLeft{
    % Ubah nilai berikut jika posisi horizontal background tidak sesuai
    \hspace{-4mm}

    % Ubah nilai berikut jika posisi vertikal background tidak sesuai
    \raisebox{0mm}{
      \includegraphics[width=\paperwidth,height=\paperheight]{Init/gambar/sampul-luar-tipis.png}
    }
  }
}

% Menyembunyikan nomor halaman
\thispagestyle{empty}

% Pengaturan margin untuk menyesuaikan konten sampul
\newgeometry{
  top=65mm,
  left=30mm,
  right=30mm,
  bottom=20mm
}

\begin{flushleft}

  % Pemilihan font sans serif
  \sffamily

  % Pemilihan font bold
  \fontseries{bx}
  \selectfont
  \begin{spacing}{1.5}
   \begin{large}
  FINAL PROJECT - \coursecode{}
\end{large}

\vspace{\fill}

\begin{spacing}{1.5}
  \begin{Large}
    \engtatitle{}
  \end{Large}
\end{spacing}

\vspace{\fill}

\begin{large}
  \name{} \\
  \textmd{NRP \nrp{}}
\end{large}

\vspace{\fill}

\begin{large}
  \textmd{Advisor} \\
  \advisor{} \\
  \textmd{NIP \advisornip{}} \\
  \coadvisor{} \\
  \textmd{NIP \coadvisornip{}}
\end{large}

\vspace{\fill}

Undergraduate Study Program of \engstudyprogram{} \\

\mdseries

Department of \engdepartment{} \\
Faculty of \engfaculty{} \\
Sepuluh Nopember Institute of Technology

\place{} \\ \the\year{}

  \end{spacing}

\end{flushleft}

\restoregeometry

\cleardoublepage

% Label tabel dan gambar dalam bahasa indonesia
\renewcommand{\figurename}{Gambar}
\renewcommand{\tablename}{Tabel}

% Pengaturan ukuran indentasi paragraf
\setlength{\parindent}{2em}

% Pengaturan ukuran spasi paragraf
\setlength{\parskip}{1ex}

% Lembar pengesahan
\input{Init/lembar-pengesahan.tex}
\cleardoublepage
\input{Init/lembar-pengesahan-en.tex}
\cleardoublepage

% Pernyataan keaslian
\input{Init/pernyataan-keaslian.tex}
\cleardoublepage
\input{Init/pernyataan-keaslian-en.tex}
\cleardoublepage

% Nomor halaman pembuka dimulai dari sini
\pagenumbering{roman}

% Abstrak Bahasa Indonesia
\begin{center}
  \large\textbf{ABSTRAK}
\end{center}

\addcontentsline{toc}{chapter}{ABSTRAK}

\vspace{2ex}

\begingroup
% Menghilangkan padding
\setlength{\tabcolsep}{0pt}

\noindent
\begin{tabularx}{\textwidth}{l >{\centering}m{2em} X}
  Nama Mahasiswa    & : & \name{}         \\

  Judul Tugas Akhir & : & \tatitle{}      \\

  Pembimbing        & : & 1. \advisor{}   \\
                    &   & 2. \coadvisor{} \\
\end{tabularx} 
\endgroup

% Ubah paragraf berikut dengan abstrak dari tugas akhir
Sistem ini dirancang untuk melakukan estimasi kecepatan kendaraan pada NVIDIA Jetson Nano dengan menggabungkan deteksi objek menggunakan YOLOv8 yang telah dioptimasi. Algoritma pelacakan multi-objek OC-SORT yang ringan namun akurat dalam mempertahankan identitas tiap kendaraan antar frame. Kecepatan setiap kendaraan dihitung berdasarkan perpindahan koordinat dalam satuan piksel per frame. Berkat memanfaatkan TensorRT untuk inferensi dan menyusun pipeline image preprocessing yang efisien sistem mampu mencapai throughput antara 8 hingga 12 FPS saat menjalankan deteksi dan pelacakan secara simultan Dengan konfigurasi yang ringkas dan penggunaan perangkat keras yang terjangkau, solusi ini menawarkan alternatif praktis untuk penerapan pengawasan di perkotaan, tanpa memerlukan infrastruktur pencahayaan atau pemrosesan terpusat yang kompleks.


% Ubah kata-kata berikut dengan kata kunci dari tugas akhir
Kata Kunci: Pengawasan, Drone, Estimasi kecepatan, Jetson Nano, YOLOv8

\cleardoublepage

% Abstrak Bahasa Inggris
\begin{center}
  \large\textbf{ABSTRACT}
\end{center}

\addcontentsline{toc}{chapter}{ABSTRACT}

\vspace{2ex}

\begingroup
% Menghilangkan padding
\setlength{\tabcolsep}{0pt}

\noindent
\begin{tabularx}{\textwidth}{l >{\centering}m{3em} X}
  \emph{Name}     & : & \name{}         \\

  \emph{Title}    & : & \engtatitle{}   \\

  \emph{Advisors} & : & 1. \advisor{}   \\
                  &   & 2. \coadvisor{} \\
\end{tabularx}
\endgroup

% Ubah paragraf berikut dengan abstrak dari tugas akhir dalam Bahasa Inggris
\emph{
  This system is designed to estimate vehicle speed on the NVIDIA Jetson Nano by integrating optimized YOLOv8 object detection with the lightweight yet accurate OC-SORT multi-object tracking algorithm to preserve each vehicle’s identity across frames. Speed is calculated from the per-frame pixel displacement of bounding box coordinates. By leveraging TensorRT for inference and an efficient image-preprocessing pipeline, the system achieves a throughput of 8–12 FPS for simultaneous detection and tracking. With its compact configuration and cost-effective hardware, this solution offers a practical alternative for urban surveillance without requiring additional lighting infrastructure or centralized processing.
}

% Ubah kata-kata berikut dengan kata kunci dari tugas akhir dalam Bahasa Inggris
\emph{Keywords}: \emph{Traffic Control, Drone, Speed Estimation, Jetson Nano, YOLOv8}

\cleardoublepage

% Kata pengantar
\begin{center}
  \Large
  \textbf{KATA PENGANTAR}
\end{center}

\addcontentsline{toc}{chapter}{KATA PENGANTAR}

\vspace{2ex}

% Ubah paragraf-paragraf berikut dengan isi dari kata pengantar

Puji dan syukur kehadirat Tuhan Yang Maha Esa, atas segala rahmat dan karunia-Nya, sehingga penulis dapat menyelesaikan penelitian ini yang berjudul "\tatitle"

Penelitian ini disusun dalam rangka pemenuhan Tugas Akhir sebagai syarat kelulusan Mahasiswa ITS. Oleh karena itu, penulis mengucapkan banyak terima kasih kepada:

\begin{enumerate}[nolistsep]
  \item Bapak \advisor, selaku Kepala Departemen Teknik Komputer, Fakultas Teknologi Elektro dan Informatika Cerdas, Institut Teknologi Sepuluh Nopember yang juga menjadi Dosen Pembimbing I, serta Ibu \coadvisor, selaku Dosen Pembimbing II yang telah memberikan masukan dan arahan kepada penulis selama pengerjaan tugas akhir ini.
  \item Bapak-Ibu dosen pengajar Departemen Teknik Komputer, atas pelajaran dan ilmu yang telah diberikan kepada penulis selama masa perkuliahan.
  \item Kedua orang tua, kakak, dan seluruh keluarga tersayang yang telah membantu dalam segi spiritual dan materiil.
  \item Harist, Bram, dan teman-teman yang lain baik dari jurusan Teknik Komputer atau bukan, telah memberikan dukungan, motivasi, dan bantuan kepada penulis selama pengerjaan tugas akhir ini.
\end{enumerate}

Penelitian ini masih jauh dari kata sempurna, untuk itu penulis mengharapkan dengan segenap hati atas saran dan kritik yang membangun. Akhir kata, semoga penelitian ini dapat memberikan banyak manfaat untuk pembaca dan banyak pihak lainnya.

\begin{flushright}
  \begin{tabular}[b]{c}
    \place{}, \MONTH{} \the\year{} \\
    \\
    \\
    \\
    \\
    \name{}
  \end{tabular}
\end{flushright}

\cleardoublepage

% Daftar isi
\renewcommand*\contentsname{DAFTAR ISI}
\addcontentsline{toc}{chapter}{\contentsname}
\tableofcontents
\cleardoublepage

% Daftar gambar
\renewcommand*\listfigurename{DAFTAR GAMBAR}
\addcontentsline{toc}{chapter}{\listfigurename}
\listoffigures
\cleardoublepage

% Daftar tabel
\renewcommand*\listtablename{DAFTAR TABEL}
\addcontentsline{toc}{chapter}{\listtablename}
\listoftables
\cleardoublepage

% Nomor halaman isi dimulai dari sini
\pagenumbering{arabic} 

% Bab 1 pendahuluan

\begin{spacing}{1.2}
  \chapter{PENDAHULUAN}
\end{spacing}


\pagenumbering{arabic}
\vspace{4ex}


\section{Latar Belakang}
LaTeX adalah sistem penyusunan huruf yang berkualitas tinggi; itu termasuk fitur yang dirancang untuk produksi dokumentasi teknis dan ilmiah. LaTeX adalah standar de facto untuk komunikasi dan publikasi mm,<<,,dokumen ilmiah. LaTeX tersedia sebagai perangkat lunak gratis.

\section{Rumusan Masalah}
Bagian ini untuk menulis rumusan masalah.
\section{Tujuan}
Tujuan dari tutorial ini adalah \cite{Koza1996}
\begin{enumerate}
	\item Tujuan Pertama
	\item Tujuan Kedua
\end{enumerate}
\section{Batasan Masalah}
Tutorial ini dibatasi pada penggunaan Latex untuk penulisan tesis. 
\section{Manfaat}
Diharapkan mahasiswa dapat mudah menulis tesis sehingga dapat cepat menyelesaikan studi di Magister Teknik Elektro ITS.
\subsection{Contoh Subseksi }
\subsubsection{Contoh SubSub Seksi}

\begin{equation}
y=cos(\alpha x)
\end{equation}

\cleardoublepage

\begin{spacing}{1.2}
	\chapter{TINJAUAN PUSTAKA}
\end{spacing}
  
\vspace{4ex}

\section{Hasil Penelitian Terdahulu}
Pengerjaan penelitian ini juga dipengaruhi adanya beberapa hasil penelitian terdahulu yang terkait, sebagai berikut:

\subsection{Perhitungan Kecepatan Kendaraan Menggunakan Drone Bergerak dengan Metode Deep Learning}
\label{subsec:Iqbal2024}
Penelitian ini dikembangkan oleh Iqbal Fatchurozi, untuk penyelesaian Tugas Akhir. Penulis melakukan penelitian terkait penggunaan \emph{drone} dalam pengawasan lalu lintas. Implementasi pengolahan citra video untuk melakukan perhitungan kecepatan kendaraan dari atas menggunakan \emph{drone} yang dilakukan dengan komputer untuk komputasinya. Metode deteksi objek yang digunakan adalah YOLOv8 dengan menggunakan streamlit untuk mengirimkan video ke komputer server sebagai tempat pemrosesan data \cite{Iqbal2024}. 

\subsection{Vehicle Tracking and Speed Estimation from Unmanned Aerial Vehicles Using Segmentation-Initialised Trackers}
\label{subsec:Tilon2023}
Pendekatan umum dalam estimasi kecepatan kendaraan berbasis \emph{Unmanned Aerial Vehicle} (UAV) menggunakan deteksi objek seperti \emph{YOLOv4} yang dikombinasikan dengan pelacak seperti \emph{DeepSORT}. Meskipun akurat, metode ini kurang efisien untuk \emph{edge device} karena beban komputasi yang tinggi. Sebagai alternatif, digunakan pelacak \emph{MOSSE} yang ringan, sementara peneliti yang lain menunjukkan pentingnya jarak UAV terhadap objek dalam sistem pelacakan.

Menanggapi keterbatasan tersebut, Tilon dan Nex (2023) mengembangkan metode pelacakan berbasis segmentasi menggunakan model \emph{CABiNet}. Dengan inisialisasi pelacakan dari hasil segmentasi, metode ini mampu berjalan pada \emph{edge device} seperti \emph{Jetson Xavier NX} dan menghasilkan \emph{MOTP} sebesar 0{,}872, lebih tinggi dibandingkan metode berbasis deteksi objek. Pendekatan ini menawarkan efisiensi serta fleksibilitas dalam menghasilkan informasi semantik untuk pemantauan infrastruktur \cite{Tilon2023}.

\subsection{Automatic Vehicle Speed Estimation Method for Unmanned Aerial Vehicle Images}
Penelitian yang telah dilakukan oleh Hao Long, Yi-Nung, dan rekan menghasilkan metode otomatis untuk mendeteksi dan menghasilkan estimasi kecepatan kendaraan melalui citra udara dari UAV. Metode yang digunakan meliputi trannsformasi warna dengan HSV, meminimalisir bayangan, dan memisahkan objek deteksi dari latar menggunakan perbedaan temporal. Proses estimasi kecepatan yang digunakan adalah menghitung jarak perpindahan objeknya dalam satuan piksel, yang didukung dengan lebar jalan sebagai skala konversi jarak \cite{auto-vehicle-speed-uav-img}.

\subsection{A novel vehicle tracking and speed estimation with varying UAV altitude and video resolution}
Jurnal ini membahas perihal metode deteksi kendaraan dan estimasi kecepatan menggunakan video udara dari atas dengan variasi ketinggian yang dihasilkan dari \emph{Unmanned Aerial Vehicle} beserta resolusi video yang dilakukan oleh Yuqing Chen, Dongyang Zhao, dan rekan. Metode deteksi yang digunakan adalah YOLOv3, kemudian untuk estimasi kecepatannya menggunakan pemetaan piksel ke jarak nyata secara eksponensial. Pendekatan yang dilakukan untuk estimasi kecepatan kecepatannya ialah memanfaatkan hubungan eksponensial yang dihasilkan dari \emph{fitting} data antara jarak piksel dengan jarak aktual yang didukung \emph{least square} sehingga tidak diperlukan kalibrasi kamera yangg kompleks \cite{novel-vehicle-tracking}.

\subsection{Real-Time Traffic Flow Parameter Estimation From UAV Video Based on Ensemble Classifier and Optical Flow}
Jurnal yang ditulis oleh Zhibin Li, Jinjun Tang, dan rekan berisi tentang pembahasan untuk menghitung perkiraan aliran lalu lintas seperti kecepatan, kepadatan, dan volume menggunakan video dari \emph{UAV}. Klasifikasi \emph{Haar cascade} digunakan untuk menghitung ROI \emph{Region of Interest}, kemudian CNN sebagai klasifikasi akhir dalam deteksi kendaraan. Estimasi \emph{motion} dilakukan dengan \emph{optical flow} untuk mengukur perpindahan kendaraan dan latar belakang secara terpisah. Lalu, untuk memperkirakan parameter aliran lalu lintas digunakan informasi dari deteksi dan estimasi \emph{motion} untuk menghitung parameternya seperti kecepatan, kepadata, dan volume \cite{realtime-trafficflow-estimation}.

\subsection{AI-Powered Automated Road Damage Detection Using UAV Images and Deep Learning}
Penelitian ini mengembangkan sistem deteksi kerusakan jalan otomatis dengan memanfaatkan \emph{Unmanned Aerial Vehicle} (UAV) dan model deep learning berbasis CNN, khususnya varian YOLO (v5, v7, v8). Citra udara resolusi tinggi yang diambil dari berbagai sudut dan ketinggian diproses lebih dulu, meliputi pengurangan noise, dan peningkatan kontras sebelum dianalisis oleh model yang dievaluasi menggunakan metrik mAP, presisi, dan recall untuk memastikan keakuratannya. Hasil pengujian menunjukkan YOLOv8 unggul dalam akurasi dan kecepatan inferensi secara \emph{real time}, dan hasil deteksi langsung diintegrasikan ke dalam sistem GIS untuk mempermudah pemetaan serta penentuan prioritas perbaikan. Dengan demikian, pendekatan ini mempercepat inspeksi, meminimalkan kesalahan manusia, dan memungkinkan pemantauan skala besar secara berkelanjutan, sehingga perbaikan dapat dilakukan tepat waktu dan umur infrastruktur jalan pun terjaga. Untuk kedepannya, cakupan penelitian akan diperluas dengan menambah dataset, menerapkan transfer learning, serta menggunakan metode \emph{ensemble} untuk meningkatkan daya tahan dan akurasi model \cite{bujji-ai}.

\subsection{Efficient Roundabout Supervision: Real-Time Vehicle Detection and Tracking on Nvidia Jetson Nano}
Penelitian ini dikembangkan dengan menggunakan \emph{edge device} yaitu Jetson Nano, dimana pendekatan deteksi kendaraan yang digunakan adalah \emph{YOLOv7-tiny} serta \emph{Deep SORT} untuk pelacakan. Imane Elmanaa yang merupakan penulis dari jurnal ini membahas perihal pendeteksian, pelacakan, perhitungan berbagai jenis kendaraan secara \emph{real-time} pada persimpangan di negara Maroko guna mendukung perencanaan infrastruktur \cite{efficient-roundabout-supervision}.

\subsection{Perancangan Sistem Pengukur Kecepatan Kendaraan Berbasis Kamera Menggunakan Algoritma YOLO}
Penelitian yang dilakukan oleh Zikri Giarida dan Perani Rosyani membahas perancangan sistem untuk mengukur kecepatan kendaraan berbasis kamera menggunakan YOLO. YOLO disini digunakan untuk mendeteksi kendaraan yang lewat. Proses kalibrasi dilakukan dengan menentukan koordinat dari video yang disesuaikan dengan jarak aktual di dunia nyata dengan garis pengukuran. Kemudian, kecepatan kendaraan dihitung berdasarkan waktu yang dibutuhkan untuk melewati area pengukuran dengan menggunakan rumus jarak \emph{euclidean} yang dikonversikan ke kilometer per jam \cite{perancangan-sistem-pengukur-kecepatan}.

\section{Landasan Teori}
Konsep dasar atau teori yang digunakan dalam penelitian ini tercantum pada sub-bab dibawah ini:

\subsection{DJI Phantom 4 Pro}
DJI Phantom 4 Pro merupakan salah satu model \emph{drone} keluaran DJI, yang merupakan perusahaan besar dimana berfokus pada pengembangan alat teknologi, salah satunya yang terkenal adalah produk \emph{drone}nya. Phantom 4 Pro memiliki kelebihan dibanding versi sebelumnya dimana dia sudah mampu menghindari tabrakan karena adanya \emph{rear-facing obstacle sensing system} di bagian belakang, serta sensor inframerah di bagian kiri dan kanan. \emph{Drone} ini memiliki resistansi terhadap angin dengan maksimum sebesar 10m/s, dan dapat diterbangkan selama kurang lebih 30 menit \cite{djiphantom4pro}.

Kamera gimbal yang digunakan pada \emph{drone} ini memiliki piksel sebesar 20MP. Maksimum kualitas yang dapat diatur pada kamera ini sebesar 4K 60p (H.264) dan 4K 30p (H.265) dengan maksimal \emph{bitrate} sebesar 100Mbps. Hasil rekaman maupun tangkapan layar dari kamera DJI Phantom 4 Pro dapat disimpan dengan \emph{Micro SD Card} yang dihubungkan ke \emph{drone}. Terdapat juga \emph{controller} yang dapat dihubungkan dengan USB atau HDMI melalui perangkat keras seperti, \emph{smartphone} \cite{djiphantom4pro}. 

\subsection{Nvidia Jetson Nano Developer Kit}
Nvidia Jetson Nano merupakan perangkat keras yang digunakan untuk komputasi berukuran kecil bertujuan untuk aplikasi kecerdasan buatan dan \emph{machine learning}. Jetson Nano didukung dengan GPU Maxwell 128-core, CPU quad-core ARM A57, dan Memori 4GB LPDDR4 sehingga sudah mampu melakukan pemrosesan data secara efisien. Perangkat keras ini diperuntukkan untuk aplikasi di lapangan sehingga Jetson Nano ini terbilang perangkar \emph{portable} dapat dibawa kemana saja. Komputasi dapat langsung diproses di perangkat tanpa koneksi internet terlebih dahulu sehingga memungkinkan \emph{real-time} tanpa khawatir latensi yang muncul.

Banyak penelitian yang sudah dilakukan dengan Jetson Nano, dimana perangkat ini menawarkan kemampuan dalam menjalankan komputasi \emph{machine learning} dengan cepat dan akurat. Jetson Nano juga mampu melakukan komputasi pendeteksian objek dengan menggunakan model terlatih melalui \emph{framework} seperti YOLO dan Darknet. Dalam beberapa penelitian, dilakukan percobaan untuk mengetahui kinerja Jetson Nano, seperti penggunaan daya dalam menjalankan komputasi. Hasilnya menunjukkan bahwa GPU Jetson Nano bekerja hampir maksimal, dan konsumsi RAM sebanyak 2.1 GB dari setengah kapasitas memori \cite{jetson-nano}.

\subsection{\emph{Real-Time Messaging Protocol}(RTMP)}
RTMP adalah jaringan protokol yang mampu untuk mengirimkan \emph{streaming} media seperti \emph{audio}, \emph{video}, dan data dari server ke klien. Protokol ini sering digunakan untuk mengirim video dengan latensi rendah dan siaran langsung ke server media. RTMP menggunakan protokol TCP, serta menyediakan layanan multiplex pesan dua arah melalui satu saluran TCP tetap, mencakup pengiriman data dan perintah kontrol secara bersamaan. Proses koneksi RTMP diawali dengan tahapan \emph{handshake} guna memastikan kestabilan antara server dan penonton. Umumnya, keterlambatan RTMP berada pada kisaran 5 hingga 30 detik, namun ini bisa ditekan dengan penggunaan encoding H264, yang mencatatkan latensi hanya sekitar 137,48 hingga 146,02 milidetik\cite{rtmp}.

Namun, \emph{streaming} ini perlu dikonfigurasi dengan baik apabila dirasa kurang sesuai dengan kebutuhan. RTMP juga terbukti pada jurnal penelitian terkait \emph{UAV-based data streaming}, dimana terbukti bahwa dalam pengiriman pesan data dengan protokol tersebut dapat diandalkan dan cepat ke \emph{device} yang lain\cite{rtmp}.

\subsection{\emph{Deep Learning}}
\emph{Deep learning} dibangun di atas konsep \emph{artificial neural network} dengan menyusun banyak lapisan neuron yang saling terhubung untuk mempelajari representasi fitur secara hirarki. Setiap lapisan melakukan transformasi afine yang diikuti fungsi aktivasi nonlinier seperti \emph{ReLU} dan \emph{sigmoid} sehingga model mampu mengekspresikan fungsi-fungsi kompleks. Pelatihan jaringan ini memanfaatkan algoritma \emph{backpropagation} untuk menghitung gradien secara efisien, yang kemudian dioptimalkan dengan metode \emph{gradient descent} mulai dari \emph{batch} dan \emph{mini-batch} hingga varian canggih seperti \emph{momentum} dan \emph{Adam}. Untuk menjaga kestabilan dan mempercepat proses konvergensi, diaplikasikan juga teknik seperti inisialisasi bobot yang tepat, \emph{batch normalization}, \emph{dropout}, serta penjadwalan laju pembelajaran.

Selain jaringan \emph{feedforward}, berbagai arsitektur khusus dikembangkan sesuai karakteristik data. \emph{Convolutional Neural Network} (\emph{CNN}) memanfaatkan lapisan konvolusi dan pooling untuk mengekstrak pola spasial dalam citra, membangun peta fitur bertingkat sebelum tahap klasifikasi atau regresi. Sementara itu, \emph{Recurrent Neural Network} (RNN) dan varian berpintu seperti LSTM dan GRU dirancang untuk memproses data sekuensial seperti teks atau sinyal waktu dengan menyimpan jejak konteks antar langkah waktu melalui status tersembunyi. Di sisi tak terawasi, \emph{autoencoder} mengompresi input ke \emph{embedding} berdimensi rendah dan merekonstruksi kembali keluaran, berguna untuk deteksi anomali dan reduksi dimensi. Melalui TensorFlow 2 dan Keras, praktisi dapat mengombinasikan lapisan-lapisan ini untuk merancang model deep learning yang sesuai beragam aplikasi \cite{Geron2019}.

\subsection{YOLO}
\emph{You Only Look Once} atau disingkat YOLO merupakan algoritma deteksi objek yang dikenalkan oleh Joseph Redmon, Santosh Divvala, Ross Girshick, dan Ali Farhadi pada tahun 2015. Masalah deteksi objek sebagai regresi dibandingkan tugas klasifikasi dengan memisahkan \emph{bounding box} dan memperhitungkan ke setiap gambar yang terdeteksi \cite{yoloweb}. Dengan pendekatan tersebut, YOLO mampu memproses citra dengan waktu singkat sehingga ideal untuk aplikasi yang membutuhkan komputasi cepat seperti pengawasan dan pemantauan secara \emph{real-time} kondisi lalu lintas.

\begin{figure} [H] \centering
  \includegraphics[scale=0.7]{bab2/yolo.jpeg}
  \caption{\emph{Layer} pada YOLO}
  \label{fig:layeryolo}
\end{figure}

Pada Gambar \ref{fig:layeryolo} terlihat bahwa arsitektur CNN untuk pemrosesan citra. Proses dimulai dengan \emph{input layer} sebagai inputan gambar. Kemudian, \emph{convolution layer 1}, untuk ekstraksi fitur pada citra. Setelah itu, melewati \emph{max pooling layer 1} untuk mengurangi dimensi data dalam meningkatkan efisiensi program komputasi saat melakukan pemrosesan data sehingga mencegah \emph{overfitting}. Akan seterusnya mengulang \emph{layer} tersebut hingga pada tahap akhir, hasilnya berhasil menuju \emph{fullly-connected layer} yang mengkoneksikan semua neuron dengan tujuan menghasilkan keluaran di \emph{output layer} \cite{layercnn}.

\subsection{YOLOv8}
YOLOv8 merupakan versi terkini dari \emph{framework} YOLO, yang dikembangkan untuk mendeteksi objek pada sebuah data seperti gambar dan video secara \emph{real-time} dengan kecepatan tinggi. YOLOv8 dirancang untuk menyelaraskan akurasi deteksi dan kecepatan pemrosesan sehingga cocok untuk berbagai implementasi aplikasi seperti pengawasan dan pemantauan keamanan, lalu lintas, dan pemantauan lainnya yang membutuhkan deteksi objek secara cepat dan akurat.

YOLOv8 menggabungkan beberapa pendekatan yang lebih baik, antara lain \emph{Feature Pyramid Network} (FPN) untuk mengatur objek yang bervariasi, kemudian ada \emph{Path Aggregation Network} (PAN) yang membuat kualitas deteksi pada beberapa level fitur meningkat. YOLOv8 sudah mampu mengurangi redudansi dari komputasi melalui pemisahan fitur dengan \emph{backbone} CSPDarknet sehingga hasil dari pemrosesan dapat lebih efisien. Selain itu, pendekatan \emph{anchor-free} dalam \emph{Adaptive Anchor-free Head} digunakan oleh YOLOv8 guna menghilangkan \emph{anchor-box} dan meningkatkan fleksibilitas. Maka dari itu, YOLOv8 memberikan performa deteksi objek yang lebih baik secara keseluruhan \cite{yolov8}.

\subsection{TensorRT}
Implementasi TensorRT pada model YOLOv8 bertujuan untuk mengoptimalkan proses inferensi dengan mengaplikasikan berbagai teknik percepatan seperti kuantisasi, \emph{layer fusion}, dan optimasi kernel yang dirancang khusus untuk perangkat keras NVIDIA GPU. TensorRT memproses model deep learning agar dapat dijalankan dengan latensi yang rendah dan throughput tinggi tanpa mengurangi akurasi, sehingga sangat sesuai untuk aplikasi \emph{real-time}. Dalam penelitian ini, TensorRT digunakan untuk meningkatkan kecepatan inferensi pada model YOLOv8 yang telah dimodifikasi pada arsitektur \emph{Feature Pyramid Network} (FPN), sehingga dapat memenuhi kebutuhan inferensi dengan latensi yang sangat singkat.

Hasil eksperimen menunjukkan bahwa implementasi TensorRT meningkatkan kecepatan inferensi secara signifikan, dengan penurunan waktu inferensi hingga 7,1 ms dibandingkan model tanpa optimasi. Peningkatan ini disertai dengan sedikit kenaikan nilai \emph{mean Average Precision} (mAP50-95), menandakan bahwa optimasi ini tetap menjaga akurasi deteksi objek. Modifikasi arsitektur YOLOv8 pada FPN yang hanya menggunakan detection head pada skala objek tertentu, dipadukan dengan TensorRT memberikan performa terbaik dalam hal kecepatan dan presisi. Pendekatan ini mempercepat inferensi hingga mencapai sekitar 12 ms, sangat ideal untuk implementasi dalam sistem deteksi berbasis visi komputer dengan kebutuhan respons waktu nyata \cite{tensorrt}.


\subsection{\emph{Observation-Centric} SORT (OC-SORT)}
Kalman Filter adalah salah satu metode utama yang sering digunakan dalam sistem pelacakan multi-objek, termasuk dalam algoritma \emph{SORT} (\emph{Simple Online and Realtime Tracking}). Metode ini berasumsi bahwa pergerakan objek mengikuti model gerak linear dengan kecepatan tetap pada rentang waktu yang singkat. Dalam pengaplikasiannya untuk pelacakan kendaraan menggunakan drone DJI Phantom 4 Pro dan metode YOLOv8 yang dijalankan pada Jetson Nano, Kalman Filter dipakai untuk memprediksi posisi kendaraan berdasarkan kotak pembatas (\emph{bounding box}) hasil deteksi objek. Namun, metode ini memiliki keterbatasan saat objek terhalang (\emph{occlusion}) atau bergerak dengan pola non-linear, sehingga akumulasi kesalahan prediksi dapat terjadi seiring berjalannya waktu.

OC-SORT (\emph{Observation-Centric SORT}) merupakan pengembangan dari \emph{SORT} yang mengatasi kelemahan tersebut dengan pendekatan yang berfokus pada observasi nyata, bukan hanya prediksi hasil estimasi. Metode ini mengadopsi strategi \emph{Observation-Centric Re-Update} (\emph{ORU}), di mana kesalahan yang menumpuk pada Kalman Filter selama periode \emph{occlusion} diperbaiki dengan menggunakan data observasi virtual yang dibentuk melalui interpolasi dari pengamatan terakhir sebelum \emph{occlusion} dan pengamatan saat objek kembali terlihat. Dengan pendekatan ini, \emph{OC-SORT} mampu meningkatkan akurasi pelacakan objek terutama saat kendaraan yang sempat terhalang muncul kembali, serta menambahkan komponen \emph{Observation-Centric Momentum} (\emph{OCM}) untuk memperbaiki pencocokan data berdasarkan arah dan kecepatan gerak yang diukur dari observasi historis\cite{Cao2023}.

\subsection{ByteTrack}
ByteTrack adalah algoritma pelacakan \emph{Multi-Object Tracking} (MOT) yang meningkatkan akurasi dengan memanfaatkan hampir keseluruhan deteksi objek, dari deteksi dengan skor rendah dan deteksi dengan skor tinggi. Dengan metode ByteTrack, dapat mengatasi permasalahan deteksi objek yang tersembunyi dalam frame, serta fragmentasi jalur pelacakan yang sering terjadi pada objek yang tidak terdeteksi dalam frame berturut-turut. Keunggulan ByteTrack dalam pelacakan adalah melakukan asosiasi terhadap objek deteksi yang hilang dengan objek yang telah dilacak pada frame sebelumnya, dan juga ketika objek terhalang atau memiliki kualitas rendah. Sehingga, hal tersebut memungkinkan pemulihan objek yang hilang, serta meningkatkan kontinuitas pelacakan \cite{Zhang2022ByteTrack}.

\begin{figure} [H] \centering
  \includegraphics[scale=1]{bab2/mota_bytetrack.png}
  \caption{Perbandingan MOTA-IDFI-FPS terhadap \emph{tracker}}
  \label{fig:mota-tracker}
\end{figure}

Dalam grafik pada Gambar \ref{fig:mota-tracker} menunjukkan bahwa ByteTrack sebagai algoritma \emph{Multiple Object Tracking Accuracy} tertinggi dibandingkan dengan yang lain. ByteTrack sangat efisien dalam hal pelacakan sehingga FPS yang dihasilkan juga lebih tinggi meskipun tidak secepat FairMOT. Algoritma ByteTrack menggabungkan dua proses asosiasi dalam \emph{tracking} yang dilakukan. Pertama, deteksi pada skor tinggi (\emph{high-confidence}) yang memastikan pelacakan akurat pada objek yang tidak terhalang dan mencocokkan objek yang terdeteksi sebelumnya. Kedua, deteksi dengan skor rendah (\emph{low-confidence}), yang memungkinkan memulihkan objek yang sempat tidak terlacak dan terfragmentasi dalam frame sebelumnya. Penggunaan Kalman Filter dan algoritma Hungarian pada ByteTrack untuk mencocokkan deteksi baru dengan jalur \emph{tracking} yang telah ada sehingga memungkinkan \emph{tracking} yang lebih andal meskipun dalam kondisi berbeda-beda \cite{Zhang2022ByteTrack}.

\section{Ground Sampling Distance}
Ground Sampling Distance (GSD) adalah ukuran resolusi spasial citra udara yang menghubungkan satu piksel pada gambar dengan ukuran sebenarnya di permukaan tanah. Semakin kecil nilai GSD, semakin detail citra yang dihasilkan, karena setiap piksel mewakili area yang lebih kecil. Parameter ini krusial dalam fotogrametri dan aplikasi survei udara, karena akan menentukan sejauh mana objek di permukaan dapat dikenali dan diukur dengan akurat.

Secara matematis, GSD dapat dirumuskan sebagai berikut:
\begin{equation}
  \label{eq:gsd_basic}
  \mathrm{GSD}
  = \frac{H \times \mathrm{PixelSize}}{f}
\end{equation}
Di sini $H$ menyatakan ketinggian terbang (dalam meter), $f$ adalah panjang fokus kamera (dalam milimeter), dan \emph{PixelSize} adalah ukuran satu piksel pada sensor, yang dihitung dari lebar sensor $S_w$ (mm) dibagi jumlah piksel pada sumbu lebar gambar $\mathrm{ImgW}$:
\begin{equation}
  \label{eq:pixel_size}
  \mathrm{PixelSize}
  = \frac{S_w}{\mathrm{ImgW}}
\end{equation}
Dengan menggantikan \eqref{eq:pixel_size} ke dalam \eqref{eq:gsd_basic}, diperoleh:
\begin{equation}
  \label{eq:gsd_full}
  \mathrm{GSD}
  = \frac{H \times S_w}{f \times \mathrm{ImgW}}
\end{equation}

Sebagai contoh, untuk penerbangan pada ketinggian $H = 20\,$m dengan lebar sensor $S_w = 12{,}83\,$mm, fokus $f = 8{,}6\,$mm, dan lebar citra $\mathrm{ImgW} = 5472$ piksel, nilai GSD yang diperoleh adalah sekitar
\[
  \mathrm{GSD}
  = \frac{20 \times 12{,}83}{8{,}8 \times 640}
  \approx 4{,}66\ \mathrm{cm/piksel}.
\]
Angka ini menunjukkan bahwa satu piksel pada citra setara dengan area seluas kurang lebih $4{,}68\,$cm di permukaan.

Dalam praktiknya, perubahan ketinggian terbang akan memengaruhi GSD secara linier: peningkatan $H$ akan menaikkan nilai GSD, sehingga detail citra cenderung berkurang. Selain itu, pemilihan ukuran sensor dan resolusi gambar juga perlu disesuaikan dengan tujuan pemetaan. Misalnya, pada drone DJI Phantom 4 Pro, variasi ketinggian terbang antara 20-40 m dapat menghasilkan GSD antara 4-8 cm/piksel, sehingga operator dapat memilih kombinasi parameter yang optimal untuk keseimbangan antara cakupan area dan ketajaman detail objek.



\cleardoublepage
\begin{spacing}{1.2}
	\chapter{METODOLOGI}
	\label{sec:chap3_metodologi}
\end{spacing}

\vspace{4ex}

\section{Alur Kerja Sistem}
Sistem estimasi ini akan bekerja dengan menggunakan \emph{deep learning} berbasis \emph{edge device} untuk pendeteksian objek yang difokuskan pada kendaraan yang melintas, beserta pengukuran estimasi kecepatannya. Pendeteksian ini akan dikerjakan dengan menggunakan salah satu \emph{framework} model YOLOv8 yang dikonversikan menjadi format TensorRT. Pada Gambar \ref{fig:diagramsistem}, dapat diketahui diagram dari alur kerja sistem yang telah dibuat.

\begin{figure} [H] \centering
  \includegraphics[scale=0.45]{bab3/diagramsistem.jpg}
  \caption{Diagram Alur Kerja Sistem}
  \label{fig:diagramsistem}
\end{figure}

Dilakukan akuisisi data untuk mengumpulkan dan melatih data yang diperoleh untuk proses deteksi objek. Untuk lebih akurat lagi, digunakan \emph{tracking} objek supaya dapat fleksibel mengikuti gerakan kendaraan. Pengukuran estimasi kecepatan dilakukan dengan menghitung perpindahan piksel antar \emph{frame} terhadap waktu.

\section{Akuisisi Data}
Tahap pertama yang diperlukan adalah mengumpulkan data melalui rekaman citra video dari kamera yang terpasang pada \emph{drone} hingga tahap melatih data yang diperoleh. Akuisisi data ini merupakan hal krusial karena sangat memengaruhi dari keberhasilan tahap selanjutnya seperti deteksi objek, dan \emph{tracking} objek. Berikut diagram blok dari alur akuisisi data yang dapat dilihat pada Gambar \ref{fig:akuisisidata} yang diproses sedemikian rupa sehingga data dapat digunakan untuk tahap selanjutnya.

\begin{figure} [H] \centering
  \includegraphics[scale=0.35]{bab3/akuisisidata.jpg}
  \caption{Diagram Blok Akuisisi Data}
  \label{fig:akuisisidata}
\end{figure}

%Section:Akuisisi Data%
\subsection{Pengambilan Data}
Data yang diambil merupakan hasil citra dari rekaman video yang diambil menggunakan kamera pada \emph{drone} dengan posisi \emph{birds view}, yang mana citra tampak dari atas. Salah satu data yang telah diambil dapat dilihat pada Gambar \ref{fig:datacitra_siang}, \ref{fig:datacitra_malam}.

\begin{figure} [H] \centering
  \includegraphics[scale=0.075]{bab3/citrasiang.jpg}
  \caption{Hasil Citra Pengambilan Data pada Siang Hari}
  \label{fig:datacitra_siang}
\end{figure}

\begin{figure} [H] \centering
  \includegraphics[scale=0.225]{bab3/citramalam.jpg}
  \caption{Hasil Citra Pengambilan Data pada Malam Hari}
  \label{fig:datacitra_malam}
\end{figure}

Citra tersebut diperoleh dari tiap frame yang diambil pada rekaman video \emph{drone} dari atas dengan kondisi siang hari dan malam hari. Pengambilan frame ini dilakukan secara komputasi dengan menjalankan sebuah program yang dirancang untuk mengambil setiap \emph{frame} citra dari video tersebut.

\subsection{Anotasi Data}
Setelah pengambilan data, hasil citra akan diberikan anotasi. Anotasi ini diberikan hanya pada objek yang akan dideteksi saja. Anotasi dilakukan dengan menggunakan \emph{tool} yang disediakan oleh Roboflow. Hasil anotasi dapat dilihat pada Gambar \ref{fig:anotasidatasiang} dan Gambar \ref{fig:anotasidatamalam}.

\begin{figure} [H] \centering
  \includegraphics[scale=0.5]{bab3/anotasidatasiang.png}
  \caption{Citra Siang Setelah Dianotasi}
  \label{fig:anotasidatasiang}
\end{figure}

\begin{figure} [H] \centering
  \includegraphics[scale=0.5]{bab3/anotasidatamalam.png}
  \caption{Citra Malam Setelah Dianotasi}
  \label{fig:anotasidatamalam}
\end{figure}

Objek diberi anotasi antara lain, mobil, motor, dan truk. Setelah anotasi selesai pada semua citra data, maka data dapat diubah menjadi sebuah dataset yang siap dilatih sehingga membentuk model deteksi objek.

\subsection{\emph{Pre-Processing}}
Proses ini merupakan persiapan dan transformasi data yang diperlukan untuk memudahkan saat \emph{training} serta proses analisisnya. \emph{Pre-processing} yang dilakukan pada data ini antara lain, \emph{auto-orient} dan \emph{resize} menjadi 320x320. Dua langkah tersebut diperlukan untuk mendapatkan performa \emph{training} yang lebih baik dan akurat.

\subsection{\emph{Augmentation}}
Pembuatan contoh pelatihan data baru untuk model yang dijalankan dari beberapa versi augmentasi setiap citra yang menjadi dataset. \emph{Augmentation} dapat membuat dan menambah variasi baru sehingga menambah citra data yang mana akan meningkatkan keakuratan dari model deteksi objek hasil \emph{training}. Beberapa augmentasi yang dilakukan adalah variasi \emph{flip}, rotasi 90 derajat, \emph{hue}, dan kecerahan.

\subsection{\emph{Training}}
Setelah dataset menjalani \emph{pre-processing} dan \emph{augmentation}, maka akan dilakukan pelatihan model deteksi objek dengan YOLOv8 yang merupakan \emph{framework} yang dikembangkan oleh Ultralytics. Proses \emph{training} ini akan belajar memahami pola dan mengenali objek yang dideteksi pada citra data. Model akan mengalami \emph{epoch} yang dapat diatur jumlah siklus sesuai kebutuhan. \emph{Epoch} merupakan siklus yang diproses untuk melatih model dengan setiap contoh maupun sampel dataset yang telah diproses sebelumnya. Model akan belajar mengenali objek dengan mengurangi \emph{loss} yang dihasilkan, dimana \emph{loss} merupakan jumlah banyaknya perbedaan yang dihasilkan dari prediksi model dengan anotasi yang sudah diatur. 

Pelatihan dilakukan dengan jumlah epoch sebanyak 250, dan diatur menjadi \emph{half=True} (fp16). \emph{Batch size} diatur 16, dan \emph{workers} menjadi 8. Setelah didapatkan model (.pt) dari hasil pelatihan, diubah menjadi format onnx kemudian dikonversikan ke TensorRT. Proses \emph{training} ini menggunakan perangkat keras dengan spesifikasi yang dapat dilihat pada Tabel \ref{tbl:specpc}.

  \begin{table}[h!]
    \centering
    \captionof{table}{Tabel Spesifikasi Perangkat \emph{Training}}
    \label{tbl:specpc}
    \begin{tabular}{|c|c|}
      \hline
      \rowcolor[gray]{0.9}
      \textbf{\emph{Hardware}} & \textbf{Spesifikasi} \\ \hline
      CPU               & Intel Core i9 13900K \\ \hline
      GPU               & Nvidia Geforce RTX 4090 \\ \hline
      RAM               & 64 GB \\ \hline
    \end{tabular}
  \end{table}
 
\section{Deteksi Objek}
Deteksi objek dilakukan dengan menggunakan kamera \emph{drone} DJI Phantom 4 Pro yang mana sudah diatur resolusinya menjadi 720p dan 30 fps. Hasil tangkapan dari \emph{drone} akan dikirim melalui \emph{streaming} RTMP yang akan diterima oleh Jetson Nano untuk dilakukan pendeteksian objek dengan model YOLOv8 yang sudah dikonversi ke TensorRT.

Deteksi objek dilakukan pada sebuah video yang dikirim melalui \emph{stream} RTMP. Dari \emph{streaming} tersebut, akan diterima hasil tangkapan dari kamera drone yang kemudian diubah ke 360p untuk melakukan efisiensi pemrosesan oleh Jetson Nano. Lalu, beberapa objek yang menjadi fokus deteksi, antara lain mobil, motor, dan truk akan terdeteksi sesuai dengan objek yang melintas dan tertangkap kamera drone. Pendeteksian ini dijalankan dengan menggunakan model YOLOv8 yang telah dikonversi menjadi TensorRT untuk mendukung kapabilitas pendeteksian di Jetson Nano. Objek yang terdeteksi akan diberi label dan \emph{bounding box} sesuai dengan kelasnya. \emph{Bounding box} dan label ini akan menunjukkan objek apa yang ada didalam video tersebut. Pada Gambar \ref{fig:deteksi objek} dapat dilihat hasil deteksi objek yang dilakukan dengan model YOLOv8 dengan format TensorRT.

\begin{figure} [H] \centering
  \includegraphics[scale=0.3]{bab3/deteksiobjek.png}
  \caption{Hasil Deteksi Objek}
  \label{fig:deteksi objek}
\end{figure}
\vspace{-3ex}
Hasil implementasi deteksi objek pada video menggunakan model YOLOv8 yang sudah dilatih dan dikonversi menjadi TensorRT ditunjukkan pada Gambar \ref{fig:deteksi objek}. Pada Gambar \ref{fig:deteksi objek} didapatkan \emph{confidence threshold} diatas 0.5, dikarenakan deteksi objek yang dilakukan diatur untuk membaca objek yang memiliki \emph{confidence threshold} diatas 0.45 lalu memberi \emph{bounding box} pada objek yang terdeteksi.

\section{\emph{Tracking} Objek}
Pada tahap ini, objek akan dilakukan \emph{tracking} setelah terdeteksi. Untuk algoritma \emph{tracking} yang digunakan adalah OC-SORT dan ByteTrack. Dari dua metode \emph{tracking} tersebut terdapat perbedaan yang signifikan pada penelitian yang dilakukan dimana ByteTrack lebih cepat pemrosesannya daripada OC-SORT tetapi mengorbankan akurasi lebih banyak. Dikarenakan hasil \emph{tracking} dari ByteTrack kurang baik hasilnya, dimana terkadang \emph{bounding box} tidak tercetak ke semua objek deteksi, meskipun pemrosesan lebih cepat. Maka, metode OC-SORT yang digunakan dalam \emph{tracking} di penelitian ini.

Untuk perhitungan \emph{center} dalam penentuan \emph{bounding box} digunakan persamaan titik tengah dalam geometri analitik, dengan menjumlahkan koordinat pojok sehingga ditemukan titik tengah dari objek deteksi setiap \emph{frame} dan diberi ID unik beserta nama \emph{class} sesuai dengan jenis objek yang terdeteksi.

\section{Estimasi Kecepatan}
Pada proses ini, objek yang telah terdeteksi dan menghasilkan \emph{bounding box} akan diukur kecepatannya. Dilakukan perhitungan titik tengah dengan menggunakan persamaan Euclidean yang dapat dilihat pada rumus \ref{eq:euclidean}.

\begin{equation}
  \label{eq:euclidean}
  \text{Jarak} = \sqrt{(x_2 - x_1)^2 + (y_2 - y_1)^2}
\end{equation}

\begin{flushleft}
Dimana:
\begin{align*}
x_1 & : \text{Koordinat $x$ titik pertama} \\
y_1 & : \text{Koordinat $y$ titik pertama} \\
x_2 & : \text{Koordinat $x$ titik kedua} \\
y_2 & : \text{Koordinat $y$ titik kedua} \\
\text{Jarak} & : \text{Jarak Euclidean antara dua titik}
\end{align*}
\end{flushleft}

Kemudian objek yang terdeteksi akan bergerak sehingga perpindahannya dihitung dengan selisih titik tengah. Hasil selisih tersebut akan dilakukan pembagian dengan \emph{Pixels per Meter} (PPM) yang didapatkan dari Persamaan \ref{eq:gsd_full}, setelah itu dikonversikan dimana hasilnya dari yang berbentuk cm/piksel menjadi piksel per meter. Berikut adalah hasil skala yang didapatkan untuk setiap ketinggian yang dijadikan variabel penelitian. 

\begin{table}[h!]
  \centering
  \captionof{table}{Hasil Skala Meter per Piksel}
  \label{tbl:skala_ppm}
  \begin{tabular}{|c|c|}
    \hline
    \rowcolor[gray]{0.9}
    \textbf{Ketinggian (m)} & \textbf{Meter per Piksel} \\ \hline
    20                      & 1 : 21,44                 \\ \hline
    30                      & 1 : 14,30                 \\ \hline
    40                      & 1 : 10,72                 \\ \hline
  \end{tabular}
\end{table}

Setelah itu, dilakukan perhitungan estimasi kecepatan dengan persamaan \ref{eq:speed_ms}.

\begin{equation}
  \label{eq:speed_ms}
  \text{Kecepatan(m/s)} = \frac{\text{Jarak(m)}}{\text{Waktu(s)}} \times 3{,}6
\end{equation}

\begin{flushleft}
Dimana:
\begin{align*}
Jarak & : \text{Jarak perpindahan objek deteksi} \\
Waktu & : \text{Waktu yang dibutuhkan objek deteksi berpindah} \\
\text{Kecepatan} & : \text{Hasil perhitungan kecepatan dalam satuan (m/s)}
\end{align*}
\end{flushleft}

Dikarenakan kecepatan kendaraan yang tertera pada \emph{speedometer} menggunakan satuan (\emph{km/h}) sehingga diperlukan untuk diubah satuan nya menjadi (m/s) seperti pada persamaan \ref{eq:speed_ms}.

\section{Penggunaan Alat}
Perangkat komputasi yang digunakan adalah Jetson Nano. Jetson Nano akan dikoneksikan ke \emph{controller drone} melalui protokol RTMP(\emph{Real-Time Streaming Protocol}). IP akan dikonfigurasikan antara Jetson Nano dengan \emph{drone} pada jaringan yang sama sehingga Jetson Nano dapat menerima hasil \emph{streaming}. Pembuatan \emph{server} RTMP sendiri menggunakan MediaMTX, kemudian \emph{drone} akan mengirim \emph{stream} ke \emph{server} tersebut, sehingga Jetson Nano akan membaca \emph{stream} menggunakan \emph{library} FFmpeg. Alur pengiriman data dapat dilihat pada Gambar \ref{fig:alurdata}.

\begin{figure} [H] \centering
  \includegraphics[scale=0.7]{bab3/pengirimandata.jpg}
  \caption{Diagram Alir Pengiriman Data}
  \label{fig:alurdata}
\end{figure}

Pada Gambar \ref{fig:alurdata}, inisialisasi server MediaMTX dilakukan dengan laptop dimana semua perangkat akan diatur menjadi 1 jaringan yang sama. \emph{Client} disini adalah perangkat komputasi yang digunakan, Jetson Nano.

Untuk algoritma yang digunakan dalam program estimasi ini terdapat beberapa tahapan yang harus dilewati sehingga program dapat berjalan sesuai. Diagram proses algoritma program ditunjukkan pada Gambar \ref{fig:diagramproses}.

\begin{figure} [H] \centering
  \includegraphics[scale=0.7]{bab3/algoritma.jpg}
  \caption{Diagram Alir Pemrosesan Data}
  \label{fig:diagramproses}
\end{figure}

Dalam menjalankan program estimasi ini tentu diperlukan perangkat keras untuk melakukan pemrosesannya. Perangkat keras yang digunakan dalam penelitian ini, antara lain \emph{drone}, \emph{smartphone}, laptop, Jetson Nano. Topologi perangkat keras yang digunakan dapat dilihat pada Gambar \ref{fig:topologihardware}.

\begin{figure} [H] \centering
  \includegraphics[scale=0.5]{bab3/topologi-hardware.jpg}
  \caption{Topologi Perangkat Keras}
  \label{fig:topologihardware}
\end{figure}

Pada Gambar \ref{fig:topologihardware} dapat dilihat bahwa RTMP sangat berperan penting dalam menghubungkan \emph{drone} dengan Jetson Nano sehingga hasil video yang direkam oleh \emph{drone} dapat dibaca langsung oleh Jetson Nano, dan dikomputasi juga pada Jetson Nano. Perangkat ini dibuat supaya komputasi dan visualisasi dapat dimodifikasi langsung di tempat sehingga lebih \emph{portable}.






\cleardoublepage
\chapter{PENGUJIAN}
\label{sec:chap4_pengujian}
\vspace{1ex}

\section*{}
Berbagai metodologi yang diterapkan.
\section{Pengujian Terhadap Gausian Noise}
\lipsum[1]
\section{Pengujian Terhadap dst...}
\subsection{dst1..}
\lipsum[2]
\subsection{dst2}
\lipsum[3]


\cleardoublepage
\begin{spacing}{1.2}
	\chapter{PENUTUP}
	\label{sec:chap5_penutup}
\end{spacing}
\vspace{4ex}

\section{Kesimpulan}
\label{sec:sec4_kesimpulan}
\vspace{1ex}
Berdasarkan hasil pengujian dan analisis yang telah dilakukan didapatkan beberapa kesimpulan, antara lain:
\begin{enumerate}
    \item Sistem estimasi kecepatan telah berhasil menjalankan deteksi objeknya dengan \emph{confidence threshold} diatas 0,5 untuk setiap \emph{class}, yaitu motor, mobil, dan truk.
    \item \emph{Tracking} masih dapat disempurnakan lagi atau menggunakan metode yang lain sehingga \emph{bounding box} tetap menempel terus pada objek deteksi.
    \item Sistem lebih optimal saat dijalankan di ketinggian 20m. Untuk 30m dan 40m sebenarnya bisa lebih optimal tetapi terhalang dengan spesifikasi perangkat dan juga metode deteksi (YOLOv8) yang digunakan cukup memakan memori.
    \item Deteksi malam hari berhasil dengan pencahayaan yang minimal tetapi perlu disempurnakan di bagian \emph{tracking}.
    \item Akurasi terendah didapatkan saat pengujian kecepatan 20km/h pada siang hari dengan ketinggian 30m. Untuk akurasi tertinggi diperoleh saat pengujian malam hari dengan ketinggian 20m, kecepatan 40km/h.
    \item Sistem ini dapat berjalan dimana saja dan mampu mendapatkan data nya di saat itu juga meskipun ada \emph{delay}. Implementasinya hanya memerlukan laptop, \emph{handphone}, dan internet yang tidak menguras kuota banyak saat menjalankan server satu jaringan. Tetapi, tidak bisa dijalankan terlalu lama karena terhalang perangkat komputasinya, Jetson Nano
\end{enumerate}
\section{Saran}
\label{sec:sec4_saran}
\vspace{1ex}
\begin{enumerate}
    \item Model deteksi harus diperbaiki lagi karena masih bisa disempurnakan melalui variasi akuisisi data.
    \item Menggunakan metode yang lebih ringan daripada YOLOv8 karena Jetson Nano sebenarnya sudah tidak terlalu kompatibel dengan Ultralytics versi 8 keatas.
    \item Dibutuhkan sistem pengiriman data atau \emph{streaming} yang lebih baik lagi dan mendukung untuk perangkat dengan spesifikasi rendah.
    \item Diperlukan pengujian yang lebih siap dan matang supaya tidak terjadi ketimpangan yang tidak diharapkan
\end{enumerate}


\cleardoublepage


\bibliographystyle{unsrt}
\renewcommand{\bibname}{DAFTAR PUSTAKA}
\addcontentsline{toc}{chapter}{DAFTAR PUSTAKA}

\vspace{2ex}
\bibliography{pustaka/pustaka.bib}
\newpage
\cleardoublepage
\begin{center}
  \Large
  \textbf{BIOGRAFI PENULIS}
\end{center}

\addcontentsline{toc}{chapter}{BIOGRAFI PENULIS}

\vspace{2ex}

\begin{wrapfigure}{L}{0.3\textwidth}
  \centering
  \vspace{-3ex}
  % Ubah file gambar berikut dengan file foto dari mahasiswa
  \includegraphics[width=0.3\textwidth]{BiografiPenulis/fotoku.png}
  \vspace{-4ex}
\end{wrapfigure}

% Ubah kalimat berikut dengan biografi dari mahasiswa
\name{}, lahir di Surabaya pada 11 Juni 2003. Setelah menamatkan pendidikan SMA pada tahun 2021. Melanjutkan pendidikan kuliah S-1 di Institut Teknologi Sepuluh Nopember Surabaya, Jurusan Teknik Komputer. Selain kuliah, penulis sering mengikuti kepanitiaan dan organisasi kampus. Minat terhadap pengembangan teknologi yang membuat penulis terpikirkan untuk kuliah di Teknik Komputer. Selain teknologi, manajemen dan bisnis merupakan minat lain yang sedang ditekuni.

Pembaca yang memiliki saran dan kritik terkait penelitian ini dapat menghubungi email syahrulfa1106@gmail.com

\cleardoublepage
\end{document}
