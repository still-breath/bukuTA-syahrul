\begin{center}
  \Large
  \textbf{KATA PENGANTAR}
\end{center}

\addcontentsline{toc}{chapter}{KATA PENGANTAR}

\vspace{2ex}

% Ubah paragraf-paragraf berikut dengan isi dari kata pengantar

Puji dan syukur kehadirat Tuhan Yang Maha Esa, atas segala rahmat dan karunia-Nya, sehingga penulis dapat menyelesaikan penelitian ini yang berjudul "\tatitle"

Penelitian ini disusun dalam rangka pemenuhan Tugas Akhir sebagai syarat kelulusan Mahasiswa ITS. Oleh karena itu, penulis mengucapkan banyak terima kasih kepada:

\begin{enumerate}[nolistsep]
  \item Bapak \advisor, selaku Kepala Departemen Teknik Komputer, Fakultas Teknologi Elektro dan Informatika Cerdas, Institut Teknologi Sepuluh Nopember yang juga menjadi Dosen Pembimbing I, serta Ibu \coadvisor, selaku Dosen Pembimbing II yang telah memberikan masukan dan arahan kepada penulis selama pengerjaan tugas akhir ini.
  \item Bapak-Ibu dosen pengajar Departemen Teknik Komputer, atas pelajaran dan ilmu yang telah diberikan kepada penulis selama masa perkuliahan.
  \item Kedua orang tua, kakak, keluarga, dan seorang perempuan terdekat yang tidak bisa disebut namanya yang selalu menemani dan memberi semangat dalam pengerjaan penulisan ini.
  \item Harist Ahmad Farhan, dan teman-teman yang lain, baik dari jurusan Teknik Komputer atau bukan, yang telah memberikan dukungan, motivasi, dan bantuan kepada penulis selama pengerjaan tugas akhir ini.
\end{enumerate}

Penelitian ini masih jauh dari kata sempurna, untuk itu penulis mengharapkan dengan segenap hati atas saran dan kritik yang membangun. Akhir kata, semoga penelitian ini dapat memberikan banyak manfaat untuk pembaca dan banyak pihak lainnya.

\begin{flushright}
  \begin{tabular}[b]{c}
    \place{}, \MONTH{} \the\year{} \\
    \\
    \\
    \\
    \\
    \name{}
  \end{tabular}
\end{flushright}
