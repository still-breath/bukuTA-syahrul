% Pengaturan ukuran halaman dan margin
\usepackage[a4paper,top=30mm,left=30mm,right=20mm,bottom=25mm]{geometry}
\usepackage[english]{babel}
\usepackage{xtab}
\usepackage[utf8]{inputenc}
\usepackage{amsmath}
\usepackage{multirow} % for multirow command
% Pengaturan ukuran spasi
\usepackage[singlespacing]{setspace}
\usepackage{rotating} % for sidewaystable environment
\usepackage[ruled,vlined]{algorithm2e} % for algorithm2e environment
\usepackage{pdflscape} % for landscape environment
\usepackage{hyphenat} % for \nohyphens command
% Pengaturan detail pada file PDF
\usepackage[pdfauthor={\@author},bookmarksnumbered,pdfborder={0 0 0}]{hyperref}
% Pengaturan jenis karakter
\usepackage[utf8]{inputenc}
\usepackage{makecell}
\usepackage{amsmath}

% Pengaturan pewarnaan
\usepackage[table,xcdraw]{xcolor}

% Pengaturan kutipan artikel
%\usepackage[style=ieee, backend=biber]{biblatex}

% Package lainnya
\usepackage{changepage}
\usepackage{enumitem}
\usepackage{eso-pic}
\usepackage{txfonts} % Font times
\usepackage{etoolbox}
\usepackage{graphicx}
\usepackage{lipsum}
\usepackage{caption}
\usepackage{longtable}


\usepackage{tabularx}
\usepackage{wrapfig}
\usepackage{float}
\usepackage{array} % Add this line in the preamble
\usepackage{ifthen}
\usepackage{etoolbox}
\usepackage{cite}
\usepackage{subcaption}

% Add your bibliography file
%\addbibresource{pustaka/pustaka.bib}

% Define boolean variables

%\captionsetup[longtable]{
%  width=.9\textwidth, % Adjust the width to .9 of the text width or as needed
%}

% Definisi untuk "Hati ini sengaja dikosongkan"
\patchcmd{\cleardoublepage}{\hbox{}}{
  \thispagestyle{empty}
  \vspace*{\fill}
  \begin{center}\textit{[Halaman ini sengaja dikosongkan]}\end{center}
  \vfill}{}{}

% Pengaturan penomoran halaman
\usepackage{fancyhdr}
\fancyhf{}
\renewcommand{\headrulewidth}{0pt}
\pagestyle{fancy}
\fancyfoot[LE,RO]{\thepage}
\patchcmd{\chapter}{plain}{fancy}{}{}
\patchcmd{\chapter}{empty}{plain}{}{}

% Command untuk bulan
\newcommand{\MONTH}{%
  \ifcase\the\month
  \or Januari% 1
  \or Februari% 2
  \or Maret% 3
  \or April% 4
  \or Mei% 5
  \or Juni% 6
  \or Juli% 7
  \or Agustus% 8
  \or September% 9
  \or Oktober% 10
  \or November% 11
  \or Desember% 12
  \fi}
\newcommand{\ENGMONTH}{%
  \ifcase\the\month
  \or January% 1
  \or February% 2
  \or March% 3
  \or April% 4
  \or May% 5
  \or June% 6
  \or July% 7
  \or August% 8
  \or September% 9
  \or October% 10
  \or November% 11
  \or December% 12
  \fi}

% Pengaturan format judul bab
\usepackage{titlesec}
\titleformat{\chapter}[display]{\bfseries\Large}{BAB \centering\Roman{chapter}}{0ex}{\vspace{0ex}\centering}
\titleformat{\section}{\bfseries\large}{\MakeUppercase{\thesection}}{1ex}{\vspace{1ex}}
\titleformat{\subsection}{\bfseries\large}{\MakeUppercase{\thesubsection}}{1ex}{}
\titleformat{\subsubsection}{\bfseries\large}{\MakeUppercase{\thesubsubsection}}{1ex}{}
\titlespacing{\chapter}{0ex}{0ex}{4ex}
\titlespacing{\section}{0ex}{1ex}{0ex}
\titlespacing{\subsection}{0ex}{0.5ex}{0ex}
\titlespacing{\subsubsection}{0ex}{0.5ex}{0ex}
\setcounter{secnumdepth}{3} % Untuk memberi penomoran pada \subsubsection
\newcommand{\engtatitle}{dadfa}

\newcommand{\NamaMahasiswa}[2]{
	\newcommand{\name}{#1}
	\newcommand{\nrp}{#2}

}

\newcommand{\JudulTAInd}[1]{
	\newcommand{\tatitle}{#1}
}
\newcommand{\JudulTAEng}[1]{
			\renewcommand{\engtatitle}{#1}
	}
	
\newcommand{\Tempat}[1]{
		%Surabaya
		\newcommand{\place}{#1}

	}




\newbool{bpembimbing2}
\newbool{bpenguji3}
\setbool{bpembimbing2}{false}
\setbool{bpenguji3}{false}



\newcommand{\PembimbingSatu}[2]{
\newcommand{\advisor}{#1}
\newcommand{\advisornip}{#2}
} 
\newcommand{\PembimbingDua}[2]{
\newcommand{\coadvisor}{#1}
\newcommand{\coadvisornip}{#2}
\setbool{bpembimbing2}{true}
}


\newcommand{\PengujiSatu}[2]{
\newcommand{\examinerone}{#1}
\newcommand{\examineronenip}{#2}

} 

\newcommand{\PengujiDua}[2]{
\newcommand{\examinertwo}{#1}
\newcommand{\examinertwonip}{#2}

} 

\newcommand{\PengujiTiga}[2]{
\newcommand{\examinerthree}{#1}
\newcommand{\examinerthreenip}{#2}

\setbool{bpenguji3}{true}

}

\newcommand{\KepalaDepartemen}[2]{
\newcommand{\headofdepartment}{#1}
\newcommand{\headofdepartmentnip}{#2}
} 



% jurusan\
\newcommand{\Departemen}[2]{
\newcommand{\studyprogram}{#1}
\newcommand{\engstudyprogram}{#2}
\newcommand{\department}{#1}
\newcommand{\engdepartment}{#2}
}

% fakultas
\newcommand{\Fakultas}[2]{
\newcommand{\faculty}{#1}
\newcommand{\engfaculty}{#2}
}
% singkatan fakultas
\newcommand{\SingkatanFakultas}[2]{
\newcommand{\facultyshort}{#1}
\newcommand{\engfacultyshort}{#2}
}

% departemen

% kode mata kuliah
\newcommand{\KodeMataKuliah}[1]{
\newcommand{\coursecode}{#1}
}



%\input{pustaka/variables.tex}

% Tambahkan format tanda hubung yang benar di sini
\hyphenation{
  ro-ket
  me-ngem-bang-kan
  per-hi-tu-ngan
  tek-no-lo-gi
  me-la-ku-kan
  ber-so-si-al-i-sa-si
  didu-kung
  me-na-war-kan
  seg-men-ta-tion
  meng-hu-bung-kan
}



% Pengaturan format potongan kode
\usepackage{listings}
\definecolor{comment}{RGB}{0,128,0}
\definecolor{string}{RGB}{255,0,0}
\definecolor{keyword}{RGB}{0,0,255}
\lstdefinestyle{codestyle}{
  commentstyle=\color{comment},
  stringstyle=\color{string},
  keywordstyle=\color{keyword},
  basicstyle=\footnotesize\ttfamily,
  numbers=left,
  numberstyle=\tiny,
  numbersep=5pt,
  frame=lines,
  breaklines=true,
  prebreak=\raisebox{0ex}[0ex][0ex]{\ensuremath{\hookleftarrow}},
  showstringspaces=false,
  upquote=true,
  tabsize=2,
}
\lstset{style=codestyle}
\usepackage{array}
