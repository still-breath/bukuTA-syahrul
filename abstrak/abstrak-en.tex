\begin{center}
  \large\textbf{ABSTRACT}
\end{center}

\addcontentsline{toc}{chapter}{ABSTRACT}

\vspace{2ex}

\begingroup
% Menghilangkan padding
\setlength{\tabcolsep}{0pt}

\noindent
\begin{tabularx}{\textwidth}{l >{\centering}m{3em} X}
  \emph{Name}     & : & \name{}         \\

  \emph{Title}    & : & \engtatitle{}   \\

  \emph{Advisors} & : & 1. \advisor{}   \\
                  &   & 2. \coadvisor{} \\
\end{tabularx}
\endgroup

% Ubah paragraf berikut dengan abstrak dari tugas akhir dalam Bahasa Inggris
\emph{
  This system is designed to estimate vehicle speed on the NVIDIA Jetson Nano by integrating optimized YOLOv8 object detection with the lightweight yet accurate OC-SORT multi-object tracking algorithm to preserve each vehicle’s identity across frames. Speed is calculated from the per-frame pixel displacement of bounding box coordinates. By leveraging TensorRT for inference and an efficient image-preprocessing pipeline, the system achieves a throughput of 8–12 FPS for simultaneous detection and tracking. With its compact configuration and cost-effective hardware, this solution offers a practical alternative for urban surveillance without requiring additional lighting infrastructure or centralized processing.
}

% Ubah kata-kata berikut dengan kata kunci dari tugas akhir dalam Bahasa Inggris
\emph{Keywords}: \emph{Traffic Control, Drone, Speed Estimation, Jetson Nano, YOLOv8}
