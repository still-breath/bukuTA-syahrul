\begin{center}
  \large\textbf{ABSTRAK}
\end{center}

\addcontentsline{toc}{chapter}{ABSTRAK}

\vspace{2ex}

\begingroup
% Menghilangkan padding
\setlength{\tabcolsep}{0pt}

\noindent
\begin{tabularx}{\textwidth}{l >{\centering}m{2em} X}
  Nama Mahasiswa    & : & \name{}         \\

  Judul Tugas Akhir & : & \tatitle{}      \\

  Pembimbing        & : & 1. \advisor{}   \\
                    &   & 2. \coadvisor{} \\
\end{tabularx} 
\endgroup

% Ubah paragraf berikut dengan abstrak dari tugas akhir
Sistem ini dirancang untuk melakukan estimasi kecepatan kendaraan pada NVIDIA Jetson Nano dengan menggabungkan deteksi objek menggunakan YOLOv8 yang telah dioptimasi. Algoritma pelacakan multi-objek OC-SORT yang ringan namun akurat dalam mempertahankan identitas tiap kendaraan antar frame. Kecepatan setiap kendaraan dihitung berdasarkan perpindahan koordinat dalam satuan piksel per frame. Berkat memanfaatkan TensorRT untuk inferensi dan menyusun pipeline image preprocessing yang efisien sistem mampu mencapai throughput antara 8 hingga 12 FPS saat menjalankan deteksi dan pelacakan secara simultan Dengan konfigurasi yang ringkas dan penggunaan perangkat keras yang terjangkau, solusi ini menawarkan alternatif praktis untuk penerapan pengawasan di perkotaan, tanpa memerlukan infrastruktur pencahayaan atau pemrosesan terpusat yang kompleks.


% Ubah kata-kata berikut dengan kata kunci dari tugas akhir
Kata Kunci: Pengawasan, Drone, Estimasi kecepatan, Jetson Nano, YOLOv8
