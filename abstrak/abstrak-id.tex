\begin{center}
  \large\textbf{ABSTRAK}
\end{center}

\addcontentsline{toc}{chapter}{ABSTRAK}

\vspace{2ex}

\begingroup
% Menghilangkan padding
\setlength{\tabcolsep}{0pt}

\noindent
\begin{tabularx}{\textwidth}{l >{\centering}m{2em} X}
  Nama Mahasiswa    & : & \name{}         \\

  Judul Tugas Akhir & : & \tatitle{}      \\

  Pembimbing        & : & 1. \advisor{}   \\
                    &   & 2. \coadvisor{} \\
\end{tabularx} 
\endgroup

% Ubah paragraf berikut dengan abstrak dari tugas akhir
Peningkatan populasi penduduk di Indonesia secara langsung diikuti peningkatan resiko kemacetan dan kecelakaan lalu lintas. Dari data yang diperoleh, dapat diketahui bahwa dari tahun ke tahun jumlah kendaraan bermotor di Indonesia semakin meningkat diikuti dengan angka kecelakaan lalu lintas yang terjadi, dimana salah satu faktornya adalah pelanggaran batasan kecepatan. Untuk mengatasi masalah tersebut, dikemabangkan sebuah alat pemantauan dari udara dengan pemanfaatan \emph{drone} yang dapat melakukan perhitungan estimasi kecepatan kendaraan dengan didukung teknologi pengolahan citra video beserta program komputasi dalam menghitungnya. \emph{Drone} dimanfaatkan karena kemampuannya dalam melakukan pemantauan secara langsung dari udara dengan pengontrol jarak jauh. Sistem ini juga dikembangkan dengan berbasis Jetson Nano sehingga dapat dijalankan dimana saja. Metode yang dilakukan dalam pengolahan citra video ini adalah dengan memanfaatkan \emph{YOLOv8} untuk hasil yang akurat dan tepat.

% Ubah kata-kata berikut dengan kata kunci dari tugas akhir
Kata Kunci: pengawasan, Drone, stimasi kecepatan, Jetson Nano, YOLOv8
